\documentclass[]{article}
\usepackage[letterpaper]{geometry}
\geometry{top=0.2in, left=1.0in, right=1.0in, bottom=0.5in}

%opening
\title{Speech 100}
\author{Estefany Gebert}

\begin{document}

\maketitle 


\section{Demonstration Outline}
\subsection{Topic}
\begin{itemize}
	\item Plant Propagation.
\end{itemize}
\subsection{General Purpose}
\begin{itemize}
	\item To inform.
\end{itemize}
\subsection{Purpose statement} 
\begin{itemize}
	\item To bring joy, and color to every windowsill in the city by demonstrating how to clone herbs for fresh plants all year round!
\end{itemize}
\subsection{Introduction Statement}
\begin{itemize}
	
	\item So I've always wanted to have an elaborate and colorful garden, unfortunately for me I live in a Manhattan appartment, with only one window that faces the sun less than half of the day. I still wanted to experience the joy of growig something...anything! So I though mint, because I had researched that it would be an easy herb to care for. As a beginer I thought it would be ideal!
	\item So I went to the local market that sells live plants, and fount it! Sadly a day and a half later it was dying! So, Google to the rescue! Apparently it was infected! After some more research, I found out about propagation, which seemed like the only way if the plant were to survive. 
	\item I soon put my research to practice, and success! I was able to grow a new, and healthy mint plant!
\end{itemize}

\subsection{Steps}
\begin{itemize}
	
	\item 1. First you will need a plant you want to propagate, you could buy a small one, or ask a local gardener for a cutting. 	
	\item 2. Next, you need to gather you materials. For now you will need a mall transparent container, and water (alternatively scissors). After you have succesfully rooted the cutting, you willl need a small pot and dirt.
	\item 3. Now, unless you got a cutting from a nice person who wants to share joy, you need to find your desired plant, and cut fresh growth. This means it hast to be soft and newly grown on top of the plant.
	\item 4. Then pinch off most of the leaves from the stem, leaving only the top growth. You could lso use scissors to cut the leaves off, but make sure they are clean!
	\item 6. You can now move on to fill your container in water, and put your cutting inside. Place the container on a windowsill with some light, and roots should sprout in about a week!
	\item 6. Lastly you will need a small pot, and potting soil. You can now take your plant with a thick root system, place inside the pot, hold, and cover roots with soil. 
\end{itemize}

\subsection{Conclusion Statement}
\begin{itemize}
	
	\item I have since been able to propagate more than one kind of herb, and other types of plants. 
	\item I have also shown my mom how to propagate a few plants she has in her home, and  she loves being able to always have any herb she wants. 
	\item So it is my personal belief that anyone can benefit from being able to propagate plants, because both men and wemen that enjoy cooking would yield better results from fresh herbs. 
	\item Also, if you have an annual plant, that is a plant that has a cycle of one year, you can have the plant resurge every year! Even with perenial plants, those are that live two years or more, you can still propagate ever other year. This will ensure that once that plant's cycle is over you started another plant a year before and thus it will never die! (Jokingly) Unless you forget to water it or something.
	\item You now know how to propagate plants! Anyone can enjoy! 
\end{itemize}




\end{document}
